%%%%%%%%%%%%%%%%%%%%%%%%%%%%%%%%%%%%%%%%%%%%
% LaTeX Generator 

% Créé par Arnaud Ibakuyumcu
% ENSIMAG 2A IF
%%%%%%%%%%%%%%%%%%%%%%%%%%%%%%%%%%%%%%%%%%%%

 \documentclass[a4paper,12pt]{article}

 %---Package Generique ---% 

 \usepackage[francais]{babel}
 \usepackage[utf8]{inputenc}
 \usepackage[babel=true]{csquotes}
 \usepackage{amsmath}
 \usepackage{amssymb}
 \usepackage{hyperref}
 \usepackage{float}
 \usepackage{graphicx}
 \usepackage{enumitem} %% Pour changer les puces des enum
 \graphicspath{{}}

 \usepackage{fancybox} %% Pour les encadrés : Usage \shadowbox{}
 \usepackage{fancyhdr} %% Pour les en-tetes et pieds
 \usepackage{array,multirow,makecell} %% Pour les tableaux à lignes fusionnées
 \setcellgapes{1pt}
 \makegapedcells

 %--- Partie Page de Code ---%
\usepackage{listings}
\usepackage{color}

\definecolor{dkgreen}{rgb}{0,0.6,0}
\definecolor{gray}{rgb}{0.5,0.5,0.5}
\definecolor{mauve}{rgb}{0.58,0,0.82}
%--- Definir les reglages ---%
\lstset{frame=tb,
language=C++,
aboveskip=3mm,
belowskip=3mm,
showstringspaces=false,
columns=flexible,
basicstyle={\small\ttfamily},
numbers=none,
numberstyle=\tiny\color{gray},
keywordstyle=\color{blue},
commentstyle=\color{dkgreen},
stringstyle=\color{mauve},
breaklines=true,
breakatwhitespace=true,
tabsize=3
}

 %--- Structure de la page ---%

 \topmargin -1.5 cm
 \textwidth 18 cm
 \textheight 25 cm
 \oddsidemargin -1 cm
 \marginparwidth -1 cm
 \evensidemargin -1 cm
 \parindent 0 cm

 \pagestyle{fancy}
 \renewcommand{\headrulewidth}{1pt}
 \fancyhead[C]{}
 \fancyhead[R]{Ibakuyumcu Arnaud}
 \fancyhead[L]{PEPS : Multimonde21}

 \renewcommand{\footrulewidth}{1pt}
 \fancyfoot[C]{ENSIMAG}
 \fancyfoot[L]{}
 \fancyfoot[R]{\textbf{Page : \thepage}}

 %---Raccourci Commande---%

 \newcommand{\R}{\mathbb{R}}
 \newcommand{\N}{\mathbb{N}}
 \newcommand{\A}{\mathbf{A}}
 \newcommand{\B}{\mathbf{B}}
 \newcommand{\C}{\mathbf{C}}
 \newcommand{\D}{\mathbf{D}}
 \newcommand{\ub}{\mathbf{u}}
 \newcommand{\vs}{{\bf : } \vspace*{3mm} \\}
 \newcommand{\HRule}{\rule{\linewidth}{0.5mm}} %% Pour la page de garde

 %--- Debut du document ---% 

 \begin{document}

\begin{titlepage}
\begin{center}

\includegraphics[width=0.35\textwidth]{logo}~\\[2.5cm] %% Modifier les réglages

\textsc{\LARGE ENSIMAG}\\[1.5cm]
\textsc{\Large }\\[0.5cm]

%Title
\HRule \\[0.4cm]

{\huge \bfseries PEPS : Multimonde21\\[0.5cm]
Modèles Mathématiques\\[0.4cm] }

\HRule \\[1.5cm]

\begin{minipage}{0.4\textwidth}
\begin{center} \large
\emph{IF MeQA}\\[2cm]
Ibakuyumcu Arnaud
\end{center}
\end{minipage}
\begin{minipage}{0.4\textwidth}
\end{minipage}

\vfill

\large {09/01/2017 }

\end{center}

\end{titlepage}
 \newpage

 %---Sommaire---%

 \tableofcontents %Compiler deux fois pour avoir la table des matières
 \newpage

 %--- Commencer a ecrire ici ---% 

\section{Modèle 1 : simulation inter-économie}
Ce premier modèle consiste à ne pas se préoccuper du changement de change lors du calcul du prix. \\
En effet, le pay-off du produit multi-monde ne fait intervenir que les performances d'indice, supprimant l'unité de change étrangère pour chaque indice.
\subsection{Sous-jacents et simulation}
\subsubsection{Les sous-jacents}
Les sous jacents pour ce modèle sont : 
\begin{itemize}[label=$\star$]
\item Les indices dans leur change $I_1,\ldots,I_6$
\item Les taux de change vers euro de chaque indice $X_1,\ldots,X_6$
\end{itemize}
Par convention $i=1$ correspond à la change euro (donc à l'indice Euronext) et $X_1=1$
\subsubsection{Modèle de simulation}
Dans l'univers risque neutre les sous-jacents ont la dynamique suivante : 
\begin{itemize}[label=$\star$]
\item $\left(\forall i\in\{1,\ldots,6\}\right) \frac{dI_i(t)}{I_i(t)}=r_i(t)dt + \sigma_i.dB(t)$
\item $\left(\forall i\in\{1,\ldots,6\}\right) \frac{dX_i(t)}{X_i(t)}=((r_1(t)-r_i(t))dt+\sigma^X_i.dB(t)$ avec $\sigma^X_i= \lambda_1-\lambda_i$
\end{itemize}
Où on pose : 
\begin{itemize}[label=$\bullet$]
\item $B$ un MB de dimension $12$ sous la proba risque neutre
\item $r_i$ la courbe de taux sans risque dans la change $i$
\item $\lambda_i$ la prime de risque dans l'économie $i$
\end{itemize}
\subsubsection{Estimation des paramètres}
\underline{Pour les $r_i$ :} il faut trouver un modèle déterministe pour la courbe des taux et estimer les paramètres de ce modèles. \\

\underline{Pour les $\sigma$ :} on utilise les log-rendements historiques. \\
En effet, posons $\sigma$ la matrice $12x12$ de volatilité du marché. Les modèles mis en place entrainent que le prix d'un sous-jacent est de la forme (proba historique) : 
\begin{center}
$Z_i(t+1)=Z_i(t)e^{\int\limits_t^{t+1}\mu_i(s)ds - \frac{\sigma_i^2}{2} + \sigma_i.\varepsilon(t)}$ où $\varepsilon\sim(iid)\mathcal{N}_{12}(0,1)$
\end{center}
Le log-rendement entre deux périodes est donc : $R_i(t)=ln\left[\frac{Z_i(t+1)}{Z_i(t)}\right] = \int\limits_t^{t+1}\mu_i(s)ds - \frac{\sigma_i^2}{2} + \sigma_i.\varepsilon(t)$ \\[2mm]
Avec $\sigma_i$ est la $i$-ème ligne de $\sigma$ et $\sigma_i^2=\sum\limits_{k=1}^{12}\sigma_{ik}^2$ \\[2mm]
$\Rightarrow$ En particulier on tire 
\begin{center}
$CoVar[R_i(t),R_j(t)] = (\sigma\sigma^T)_{ij}$
\end{center}
\newpage
Un estimateur classique de $(\sigma\sigma^T)_{ij}$ est donc
\begin{center}
 $\overset{\wedge}{C}_{ij}=\frac{1}{T-1}\sum\limits_{t=1}^T\left(\left(R_i(t)-\overline{R}_i\right)\left(R_j(t)-\overline{R}_j\right)\right)$ où $\overline{R}_i=\frac{1}{T}\sum\limits_{t=1}^TR_i(t)$
 \end{center}
 On obtient alors qu'un estimateur de $\sigma$ est $Cholesky(\overset{\wedge}{C})$.
\subsection{Portefeuille de couverture}
On note dans la suite $B_i$ l'actif sans risque de l'économie $i$ (c'est à dire le zéro-coupon de maturité $T$) \\[1mm]
La composition (et le prix en euro) du portefeuille de couverture est : 
\begin{center}
$P_{couv}=X_1(\Delta_{I_1}I_1)+\ldots+X_6(\Delta_{I_6}I_6)+\Delta_{B_1}(X_1B_1)+\ldots+\Delta_{B_6}(X_6B_6)$
\end{center}
\subsection{Prix du produit}
Comme évoqué plus haut, le prix ne dépend pas des devises utilisées dans l'économie de l'indice étudié. \\
Le prix est donc : 
\begin{center}
$Prix^{euro}=f(I_1(t),\ldots,I_6(t);t)$
\end{center}
\subsection{Calcul des deltas}
La couverture demande $\nabla P_{couv}=\nabla f$ soit $\left(\forall i\in\{1,\ldots,6\}\right)$ : 
\begin{center}
\begin{tabular}{cl}
& $ \begin{cases}
\frac{\partial P_{couv}}{\partial I_i} = \frac{\partial f}{\partial I_i} \\
\frac{\partial P_{couv}}{\partial X_i} = 0
\end{cases}
$ \\[2mm]
$\Leftrightarrow$ & $ \begin{cases}
X_i\Delta_{I_i} = \frac{\partial f}{\partial I_i} \\
\Delta_{I_i}I_i+B_i\Delta_{B_i} = 0
\end{cases}
$ \\[1mm]
$\Leftrightarrow$ & $ \begin{cases}
\Delta_{I_i} = \frac{1}{X_i}\frac{\partial f}{\partial I_i} \\
\Delta_{B_i} = -\frac{\Delta_{I_i}I_i}{B_i}
\end{cases}
$
\end{tabular}
\end{center}
\newpage
\section{Modèle 2 : simulation globale}
Pour ce modèle, on utilisera la simulation des indices en change domestique (euro).
\subsection{Sous-jacents et simulation}
\subsubsection{Les sous-jacents}
Les sous jacents pour ce modèle sont : 
\begin{itemize}[label=$\star$]
\item Les indices dans la change euro $Y_1=X_1I_1,\ldots,Y_6=X_6I_6$
\item Les taux de change vers euro de chaque économie $X_1,\ldots,X_6$
\end{itemize}
Par convention $i=1$ correspond à la change euro (donc à l'indice Euronext) et $X_1=1$
\subsubsection{Modèle de simulation}
Dans l'univers risque neutre les sous-jacents ont la dynamique suivante : 
\begin{itemize}[label=$\star$]
\item $\left(\forall i\in\{1,\ldots,6\}\right) \frac{dY_i(t)}{Y_i(t)}=r_1(t)dt + \sigma_i.dB(t)$ avec $\sigma_i=\sigma^X_i+\sigma^I_i$
\item $\left(\forall i\in\{1,\ldots,6\}\right) \frac{dX_i(t)}{X_i(t)}=((r_1(t)-r_i(t))dt+\sigma^X_i.dB(t)$ avec $\sigma^X_i= \lambda_1-\lambda_i$
\end{itemize}
Où on pose : 
\begin{itemize}[label=$\bullet$]
\item $B$ un MB de dimension $12$ sous la proba risque neutre
\item $r_i$ la courbe de taux sans risque dans la change $i$
\item $\lambda_i$ la prime de risque dans l'économie $i$
\end{itemize}
L'intérêt de ce modèle est que tout est simulé en proba risque neutre domestique.
\subsubsection{Estimation des paramètres}
Les paramètres s'estiment de la même façon. Bien evidemment on estimera directement $\sigma_i$ plutôt que $\sigma^X_i$ et $\sigma^I_i$ séparément. 
\subsection{Portefeuille de couverture}
Le portefeuille de couverture est bien evidemment constitué des mêmes quantités de sois-jacent que précédemment.
\subsection{Prix du produit}
Ici, le prix du produit fait intervenir les devises étrangères. Pour calculer le pay-off avec les simulations de $Y_i=X_iI_i$ on use de la formule suivante pour les performances : 
\begin{center}
$\frac{I_n-I_0}{I_0}=\frac{\frac{Y_n}{X_n}-\frac{Y_0}{X_0}}{\frac{Y_0}{X_0}}=\frac{\frac{X_0}{X_n}Y_n-Y_0}{Y_0}$
\end{center}
Ainsi, le prix est donc fonction des devises aussi : 
\begin{center}
$Prix = g(Y_1(t),\ldots,Y_n(t)\;;\;X_1(t),\ldots,X_n(t)\;;\;t)$
\end{center}
\newpage
\subsection{Calcul des deltas}
La couverture demande toujours $\nabla P_{couv}=\nabla f$ soit $\left(\forall i\in\{1,\ldots,6\}\right)$ : 
\begin{center}
\begin{tabular}{cl}
& $ \begin{cases}
\frac{\partial P_{couv}}{\partial Y_i} = \frac{\partial g}{\partial y_i} \\
\frac{\partial P_{couv}}{\partial X_i} = \frac{\partial g}{\partial x_i}
\end{cases}
$ \\[2mm]
$\Leftrightarrow$ & $ \begin{cases}
\Delta_{I_i} = \frac{\partial g}{\partial y_i} \\
\Delta_{I_i}I_i+B_i\Delta_{B_i} = \frac{\partial g}{\partial x_i}
\end{cases}
$ \\[1mm]
$\Leftrightarrow$ & $ \begin{cases}
\Delta_{I_i} = \frac{\partial g}{\partial y_i} \\
\Delta_{B_i} = \frac{\frac{\partial g}{\partial x_i}-\Delta_{I_i}I_i}{B_i}
\end{cases}
$
\end{tabular}
\end{center}
 \end{document}